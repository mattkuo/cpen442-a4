
%% bare_conf.tex
%% V1.4a
%% 2014/09/17
%% by Michael Shell
%% See:
%% http://www.michaelshell.org/
%% for current contact information.
%%
%% This is a skeleton file demonstrating the use of IEEEtran.cls
%% (requires IEEEtran.cls version 1.8a or later) with an IEEE
%% conference paper.
%%
%% Support sites:
%% http://www.michaelshell.org/tex/ieeetran/
%% http://www.ctan.org/tex-archive/macros/latex/contrib/IEEEtran/
%% and
%% http://www.ieee.org/

%%*************************************************************************
%% Legal Notice:
%% This code is offered as-is without any warranty either expressed or
%% implied; without even the implied warranty of MERCHANTABILITY or
%% FITNESS FOR A PARTICULAR PURPOSE!
%% User assumes all risk.
%% In no event shall IEEE or any contributor to this code be liable for
%% any damages or losses, including, but not limited to, incidental,
%% consequential, or any other damages, resulting from the use or misuse
%% of any information contained here.
%%
%% All comments are the opinions of their respective authors and are not
%% necessarily endorsed by the IEEE.
%%
%% This work is distributed under the LaTeX Project Public License (LPPL)
%% ( http://www.latex-project.org/ ) version 1.3, and may be freely used,
%% distributed and modified. A copy of the LPPL, version 1.3, is included
%% in the base LaTeX documentation of all distributions of LaTeX released
%% 2003/12/01 or later.
%% Retain all contribution notices and credits.
%% ** Modified files should be clearly indicated as such, including  **
%% ** renaming them and changing author support contact information. **
%%
%% File list of work: IEEEtran.cls, IEEEtran_HOWTO.pdf, bare_adv.tex,
%%                    bare_conf.tex, bare_jrnl.tex, bare_conf_compsoc.tex,
%%                    bare_jrnl_compsoc.tex, bare_jrnl_transmag.tex
%%*************************************************************************


% *** Authors should verify (and, if needed, correct) their LaTeX system  ***
% *** with the testflow diagnostic prior to trusting their LaTeX platform ***
% *** with production work. IEEE's font choices and paper sizes can       ***
% *** trigger bugs that do not appear when using other class files.       ***                          ***
% The testflow support page is at:
% http://www.michaelshell.org/tex/testflow/



\documentclass[conference]{IEEEtran}
% If IEEEtran.cls has not been installed into the LaTeX system files,
% manually specify the path to it like:
% \documentclass[conference]{../sty/IEEEtran}





% Some very useful LaTeX packages include:
% (uncomment the ones you want to load)


% *** MISC UTILITY PACKAGES ***
%
%\usepackage{ifpdf}
% Heiko Oberdiek's ifpdf.sty is very useful if you need conditional
% compilation based on whether the output is pdf or dvi.
% usage:
% \ifpdf
%   % pdf code
% \else
%   % dvi code
% \fi
% The latest version of ifpdf.sty can be obtained from:
% http://www.ctan.org/tex-archive/macros/latex/contrib/oberdiek/
% Also, note that IEEEtran.cls V1.7 and later provides a builtin
% \ifCLASSINFOpdf conditional that works the same way.
% When switching from latex to pdflatex and vice-versa, the compiler may
% have to be run twice to clear warning/error messages.






% *** CITATION PACKAGES ***
%
%\usepackage{cite}
% cite.sty was written by Donald Arseneau
% V1.6 and later of IEEEtran pre-defines the format of the cite.sty package
% \cite{} output to follow that of IEEE. Loading the cite package will
% result in citation numbers being automatically sorted and properly
% "compressed/ranged". e.g., [1], [9], [2], [7], [5], [6] without using
% cite.sty will become [1], [2], [5]--[7], [9] using cite.sty. cite.sty's
% \cite will automatically add leading space, if needed. Use cite.sty's
% noadjust option (cite.sty V3.8 and later) if you want to turn this off
% such as if a citation ever needs to be enclosed in parenthesis.
% cite.sty is already installed on most LaTeX systems. Be sure and use
% version 5.0 (2009-03-20) and later if using hyperref.sty.
% The latest version can be obtained at:
% http://www.ctan.org/tex-archive/macros/latex/contrib/cite/
% The documentation is contained in the cite.sty file itself.






% *** GRAPHICS RELATED PACKAGES ***
%
\ifCLASSINFOpdf
  % \usepackage[pdftex]{graphicx}
  % declare the path(s) where your graphic files are
  % \graphicspath{{../pdf/}{../jpeg/}}
  % and their extensions so you won't have to specify these with
  % every instance of \includegraphics
  % \DeclareGraphicsExtensions{.pdf,.jpeg,.png}
\else
  % or other class option (dvipsone, dvipdf, if not using dvips). graphicx
  % will default to the driver specified in the system graphics.cfg if no
  % driver is specified.
  % \usepackage[dvips]{graphicx}
  % declare the path(s) where your graphic files are
  % \graphicspath{{../eps/}}
  % and their extensions so you won't have to specify these with
  % every instance of \includegraphics
  % \DeclareGraphicsExtensions{.eps}
\fi
% graphicx was written by David Carlisle and Sebastian Rahtz. It is
% required if you want graphics, photos, etc. graphicx.sty is already
% installed on most LaTeX systems. The latest version and documentation
% can be obtained at:
% http://www.ctan.org/tex-archive/macros/latex/required/graphics/
% Another good source of documentation is "Using Imported Graphics in
% LaTeX2e" by Keith Reckdahl which can be found at:
% http://www.ctan.org/tex-archive/info/epslatex/
%
% latex, and pdflatex in dvi mode, support graphics in encapsulated
% postscript (.eps) format. pdflatex in pdf mode supports graphics
% in .pdf, .jpeg, .png and .mps (metapost) formats. Users should ensure
% that all non-photo figures use a vector format (.eps, .pdf, .mps) and
% not a bitmapped formats (.jpeg, .png). IEEE frowns on bitmapped formats
% which can result in "jaggedy"/blurry rendering of lines and letters as
% well as large increases in file sizes.
%
% You can find documentation about the pdfTeX application at:
% http://www.tug.org/applications/pdftex


% *** PDF, URL AND HYPERLINK PACKAGES ***
%
\usepackage{url}
% url.sty was written by Donald Arseneau. It provides better support for
% handling and breaking URLs. url.sty is already installed on most LaTeX
% systems. The latest version and documentation can be obtained at:
% http://www.ctan.org/tex-archive/macros/latex/contrib/url/
% Basically, \url{my_url_here}.


% correct bad hyphenation here
% \usepackage{biblatex}
\hyphenation{op-tical net-works semi-conduc-tor}

\bibliography{a2.bib}

\begin{document}

\title{Assignment \#4\\
\Large{CPEN 442\\}}


\author{
  \today\\\\
  \IEEEauthorblockN{blue}
  \IEEEauthorblockA{Department of Electrical and Computer Engineering\\
    University of British Columbia\\
    Vancouver, Canada
    }
}


% make the title area
\maketitle


\section{Problem \#1 - 4-digit Pin}

\subsection{Result}
  \noindent Pin: 3087
  \begin{displaymath}
  time = user + sys = 0.080 + 0.027 = 0.107s
  \end{displaymath}

\subsection{Simple Entropy}
  \begin{displaymath}
    e = log_2(10^4) \approx 13 \mbox{ bits}
  \end{displaymath}

\subsection{Methology}
It is first observed that the length of the hashes are 42 bytes. The most
credible hashing algorithm would be SHA-1 since SHA-1 hashes are 40 bytes long.
Thus it was suspected that a salt must have been appended to the hash. In this
case, it was assumed the first 2 bytes to be the salt since the first 2 bytes in
some of the hashes have altering cases.

A Python
script\footnote{\url{https://github.com/mattkuo/cpen442-a4/blob/master/q1/q1.py}}
was created that generated SHA-1 hash values of the strings 0000 to 9999
prepended with the salt. The unix command line function \texttt{time} was used
to acquire runtime length information. From this, the \texttt{user} and
\texttt{sys} time were summed to get the total CPU time.


\section{Problem \#2 - 6 Character Password}

\subsection{Result}
  \noindent Password: Ca0SJb
  \begin{displaymath}
  time = user + sys = 89m2.711s + 0.105s = 89m2.816s
  \end{displaymath}

\subsection{Simple Entropy}
  \begin{displaymath}
    e = log_2(72^6) \approx 37 \mbox{ bits}
  \end{displaymath}

\subsection{Methology}

The format of the hash is the same as problem \#1. The first two bytes of the
hash are the salt and the last 40 bytes make up the SHA-1 hash.

Hashcat, a password cracker, was used to brute force the password. A parameter
of 120 was used for the hash-type parameter with an attack mdoe of
3\footnote{Hashcat params:
\url{https://github.com/mattkuo/cpen442-a4/tree/master/q2}}.

\section{Problem \#3 - Plaintext Password}
\subsection{Result}
  \noindent Password: H9q8pR\%Oa\#\#

\subsection{Simple Entropy}
  \begin{displaymath}
    e = log_2(72^{11}) \approx 68 \mbox{ bits}\\
  \end{displaymath}

\subsection{Methology}
IDA Pro was used to reverse engineer the plain text password. It was first
observed that the program would automatically deny user access if the input was
not 11 characters long. If the password was 11 characters long, the program
would do a comparison of each character to the hardcoded paintext password and
only allow user access if every character was correct. The password was obtained
by copying 22 bytes of memory from where the program was loading the password
from. The hex bytes were then translated to ASCII characters.

If a password is incorrect, a jump instruction causes the program to display the
access denied section of the program. Thus to make the program accept any
password, one could just \texttt{nop} this jump instruction\footnote{link to
patch for \#3}.

\section{Problem \#4 - Encrypted Password}
\subsection{Result}
  \noindent Password: RFce32
  \begin{displaymath}
  time = user + sys = 392m27.809s + 0.338s = 392m28.147s
  \end{displaymath}

\subsection{Methology}
The approach used is similar to the procedure performed in problem \#3. However,
this time the program stores a SHA-1 hash instead of the password. After copying
the stored hash, an incremental attack using Hashcat was performed to acquire
the password.

A patch could be created to accept any password in exactly the same way as was
posed in problem \#3\footnote{link to patch for \#4}.

A Python script was written to patch the binary file to accept any password the
user wants by replacing the program's hash with a SHA-1 hash produced from user
input\footnote{link to python for \#4}.

\end{document}
